\documentclass[letterpaper,11pt]{article}
\usepackage{amssymb}
\usepackage{appendix}
\usepackage{eurosym}
\usepackage{geometry}
\usepackage{graphicx}
\usepackage[colorlinks,citecolor=blue]{hyperref}
\usepackage[utf8]{inputenc}
\usepackage{multirow,multicol}
\usepackage{xcolor}

% set page margin
\geometry{left=2.54cm, right=2.54cm, top=2.54cm}

% references in the format of sorted numbers
\usepackage[numbers,sort&compress]{natbib}
% references in the format of author-year
%\usepackage{natbib}
%------AASTeX format of astro journals -----------
\newcommand{\aap}{A\&A}
\newcommand{\aaps}{Astronomy \& Astrophysics Supplement}
\newcommand{\apj}{ApJ}
\newcommand{\apjs}{ApJS}
\newcommand{\araa}{ARA\&A}
\newcommand{\mnras}{MNRAS}
\newcommand{\nat}{Nature}
\newcommand{\pasj}{PASJ}
\newcommand{\procspie}{Proceedings of the SPIE}
\newcommand{\ssr}{Space Science Reviews}

\begin{document}

\begin{center}
    \LARGE{Research interest and expertise}
    \vspace{6pt}

    \Large{Junjie Mao}
\end{center}

\section*{Introduction}
Black holes, stars, and galaxies are not isolated ``islands" in the Universe. Their formation and evolution are governed by the rich dynamics of surrounding materials \cite{tum17}. For instance, to explain many key observable properties of galaxies, feedback from their member stars and supermassive black holes are required \cite{har18}. However, the details of how does it work and when does it occur remain uncertain from both an observational and theoretical perspective.
\vspace{6pt}

About half of the baryonic matter in the entire Universe is expected to be in the form of diffuse hot plasmas with a temperature up to billions of degrees. Since the hot plasmas shine in the X-ray wavelength range ($0.1-100$~\AA), X-ray spectroscopy is thus one of the main techniques to reveal their nature. Plasma diagnostics can quantify the physical properties (temperature, density, chemical composition, microscopic turbulence, line of sight velocity, etc.) of the observing targets. Such valuable information is essential to understand the complex interplay that connects those seemingly isolated ``islands" in the Universe.
\vspace{6pt}

High-quality X-ray spectra collected by the current (\textit{XMM-Newton} and \textit{Chandra}) and future (\textit{XRISM} \cite{tas18}, \textit{Athena} \cite{nan13,bar18}, \textit{Arcus} \cite{smi16}, and \textit{HUBS} \cite{cui20}) missions also challenge plasma models widely used in the community. This requires the continuous development of plasma codes.
\vspace{6pt}

My research interests revolve around the astrophysics of the hot plasmas. In the past few years, I have gained expertise in high-resolution X-ray spectral analysis and atomic calculation for astrophysical plasma modeling \cite{mao19a,mao19b,mao19c}.
\vspace{6pt}

%\newpage
% shrink the space between references
\setlength{\bibsep}{2pt plus 5ex}
\begin{multicols}{2}
\small
\begin{thebibliography}{}
\bibitem[Tumlinson et al.(2017)]{tum17} Tumlinson, J., Peeples, M.~S., \& Werk, J.~K.\ 2017, \araa, 55, 389

\bibitem[Harrison et al.(2018)]{har18} Harrison, C.~M., Costa, T., Tadhunter, C.~N., et al.\ 2018, Nature Astronomy, 2, 198

\bibitem[Tashiro et al.(2018)]{tas18} Tashiro, M., Maejima, H., Toda, K., et al.\ 2018, \procspie, 1069922

\bibitem[Nandra et al.(2013)]{nan13} Nandra, K., Barret, D., Barcons, X., et al.\ 2013, arXiv e-prints, arXiv:1306.2307

\bibitem[Barret et al.(2018)]{bar18} Barret, D., Lam Trong, T., den Herder, J.-W., et al.\ 2018, \procspie, 106991G

\bibitem[Smith et al.(2016)]{smi16} Smith, R.~K., Abraham, M.~H., Allured, R., et al.\ 2016, \procspie, 99054M

\bibitem[Cui et al.(2020)]{cui20} Cui, W., Chen, L.-B., Gao, B., et al.\ 2020, Journal of Low Temperature Physics, doi:10.1007/s10909-019-02279-3

\bibitem[Mao et al.(2019)]{mao19a} Mao, J., de Plaa, J., Kaastra, J.~S., et al.\ 2019, \aap, 621, A9

\bibitem[Mao et al.(2019)]{mao19b} Mao, J., Mehdipour, M., Kaastra, J.~S., et al.\ 2019, \aap, 621, A99

\bibitem[Mao et al.(2019)]{mao19c} Mao, J., Kaastra, J.~S., Guainazzi, M., et al.\ 2019, \aap, 625, A122


\end{thebibliography}
\end{multicols}


\end{document}
